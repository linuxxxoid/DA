\section{Описание} 
Реализация курсового проекта сводится к следующим задачам:
\begin{enumerate}
    \item Изучить теоретическую часть сортировки слиянием.
    \item Реализовать алгоритм многопоточной сортировки.
    \item Учесть нюансы и обработать исключения.
\end{enumerate}
\pagebreak

\section{Теоретическая часть}
Сортировка слиянием (англ. Merge sort) — алгоритм сортировки, использующий O(n) дополнительной памяти и работающий за O(nlog(n)) времени.
\\
Алгоритм использует принцип «разделяй и властвуй»: задача разбивается на подзадачи меньшего размера, которые решаются по отдельности, после чего их решения комбинируются для получения решения исходной задачи. Конкретно процедуру сортировки слиянием можно описать следующим образом:
\begin{enumerate}
    \item Если в рассматриваемом массиве один элемент, то он уже отсортирован — алгоритм завершает работу.
    \item Иначе массив разбивается на две части, которые сортируются рекурсивно.
    \item После сортировки двух частей массива к ним применяется процедура слияния, которая по двум отсортированным частям получает исходный отсортированный массив.
\end{enumerate}

\\В моем задании ясно говорится, что мне требуется именно многопоточная сортировка слиянием. Благодаря тому, что сортировка слиянием построена на принципе "Разделяй и властвуй", выполнение данного алгоритма можно весьма эффективно распараллелить. При оценке асимптотики допускается, что возможен запуск неограниченного количества независимых процессов, что на практике не достижимо. Более того, при реализации имеет смысл ограничить количество параллельных потоков, что я и сделала. Для ограничения использовала функцию, которая возвращает число одновременно выполняемых потоков, то есть количество ядер процессора, и брала от этого числа логарифм степени 2. На моей машине ограничение потоков равно $log_{2}$4 = 2.
\\Был применен алгоритм рекурсивного слияния массивов T[left1...right1] и T[left2...right2]в массив A[left3...right3].
\begin{enumerate}
    \item Проверяем, что размер первого массива больше размера второго. Если это не так, то меняем местами левую позицию первого массива с левой позицией второго, аналогично для правых позиций.
    \item Вычисляем середину первого массива mid1.
    \item С помощью бинарного поиска находим середину второго массива mid2, причем T[mid2] < T[mid1].
    \item Вычисляем середину mid3 результирующего массива A с помощью формулы mid3 = left3 + (mid1 - left1) + (mid2 - left2)
    \item В результирующем массиве позиции mid3 присваиваем значение T[mid1].
    \item Сливаем T[left1...mid1 - 1] и T[left2...mid2 - 1] в A[left3...mid3 - 1]
    \item Сливаем T[mid1 + 1...right1] и T[mid2...right2] в A[mid3 + 1...right3]. 
\end{enumerate}
Оказалось, что алгоритм с применением бинарного поиска работает не так быстро, как бы хотелось, поэтому было решено модифицировать программу добавлением деления массива на 2 части по медиане. Это происходит на этапе слияния, когда на вход подается два массива с уже вычисленными левыми и правыми границами. Ищем максимум max среди двух элементов [r1, r2], так мы определим элемент, который должен стоять в конце результирующего массива, затем ищем минимум min из двух элементов [l1, l2] - это будет первым элементов в результирующем массиве. Вычисляем среднее значение min и max. Это число и будет медианой, по которой будет сливаться массив, все числа меньшие медианы будут стоять слева от нее, а большие - справа.
\\Так как существует ограничение на потоки, то при достижении этого ограничения, наш массив может быть не до конца отсортирован, поэтому нужно прибегнуть к вызову обычной последовательной сортировки слиянием.

\pagebreak

\section{Реализация}
\begin{lstlisting}[language=C]
#include <iostream>
#include <fstream>
#include <thread>
#include <vector>
#include <cmath>
#include <exception>
#include <stdexcept>
#include <algorithm>
#include <utility>
#include <chrono>


bool ReadFile(std::ifstream& file, std::vector<int64_t>& vec1) {
    if (file.is_open() && file.peek() != EOF) {
        int64_t number;
        int64_t size = 1;
        bool flag = true;
        while (size) {
            file >> number;
            if (file.eof()) { break; }
            if (flag) {
                size = number;
                flag = false;
            } else {
                vec1.emplace_back(number);
                --size;
            }
        }
    } else {
        return false;
    }

    file.close();
    return true;
}


void Reader(std::vector<int64_t>& vec1) {
    int64_t number, size;
    std::cin >> size;
    while (size) {
        std::cin >> number;
        vec1.emplace_back(number);
        --size;
    }
}


void OrdinaryMerge(std::vector<int64_t>& tmp, int64_t l1, int64_t r1, int64_t l2, int64_t r2, std::vector<int64_t>& res, int64_t lres) {
    int64_t rres = lres + (r1 - l1) + (r2 - l2) + 1;
    
    while (lres <= rres) {
        if (l1 > r1) {
            res[lres] = tmp[l2++];
        } else if (l2 > r2) {
            res[lres] = tmp[l1++];
        } else {
            res[lres] = (tmp[l1] < tmp[l2]) ? tmp[l1++] : tmp[l2++];
        }
        ++lres;
    }
}


void OrdinaryMergeSort(std::vector<int64_t>& tmp, int64_t l1, int64_t r1, std::vector<int64_t>& res) {
       if (l1 >= r1) { return; }

       int64_t mid = (l1 + r1) / 2;

       OrdinaryMergeSort(res, l1, mid, tmp);
       OrdinaryMergeSort(res, mid + 1, r1, tmp);

       OrdinaryMerge(tmp, l1, mid, mid + 1, r1, res, l1);

   }


int64_t BinarySearch(int64_t key, std::vector<int64_t>& vec, int64_t l, int64_t r) {
    r = std::max(l, r + 1);
    while (l < r ) {
        int64_t mid = (l + r) / 2;
        if (vec[mid] < key) {
            l = mid + 1;
        } else {
            r = mid;
        }
    }
    return l;
}


void MedianaSearch(int64_t key, std::vector<int64_t>& vec, int64_t l1, int64_t r1, int64_t l2, int64_t r2, std::vector<int64_t>& length) {

    uint64_t l = std::min(vec[l1], vec[l2]);
    uint64_t r = std::max(vec[r1], vec[r2]);
    int64_t infirst, insecond;
    int64_t mediana = (l + r) / 2;
    infirst = std::upper_bound(vec.begin() + l1, vec.begin() + r1, mediana) - vec.begin();
    insecond = std::upper_bound(vec.begin() + l2, vec.begin() + r2, mediana) - vec.begin();
     if (vec[infirst] > mediana && infirst != l1) {
         --infirst;
     } else if (vec[infirst] == mediana) {
         while (vec[infirst] == mediana && infirst < r1 ) {
             ++infirst;
         }
         if (vec[infirst] > mediana) {
             --infirst;
         }
     }
     if (vec[insecond] > mediana && insecond != l2) {
         --insecond;
     } else if (vec[insecond] == mediana) {
         while (vec[insecond] == mediana && insecond < r2) {
             ++insecond;
         }
         if (vec[insecond] > mediana) {
             --insecond;
         }
     }
     length.emplace_back(infirst - l1 + 1);
     length.emplace_back(infirst);
     length.emplace_back(insecond - l2 + 1);
     length.emplace_back(insecond);

    return ;
}


void Merge(std::vector<int64_t>& tmp, int64_t l1, int64_t r1, int64_t l2, int64_t r2, std::vector<int64_t>& res, int64_t lres, int64_t depth, bool limiter) {
    int64_t n1 = r1 - l1 + 1;
    int64_t n2 = r2 - l2 + 1;

    if (depth <= 0) {
        OrdinaryMerge(tmp, l1, r1, l2, r2, res, lres);
        return;
    }

    if (n1 < n2) {
        Merge(tmp, l2, r2, l1, r1, res, lres, depth, limiter);
    }
    if (n1 == 0) { return; }
    int64_t infirst, inffirst, insecond, total;
    if (limiter) {
        int64_t mid1 = (l1 + r1) / 2;
        int64_t mid2 = BinarySearch(tmp[mid1], tmp, l2, r2);
        total = lres + (mid1 - l1) + (mid2 - l2) ;

        res[total] = tmp[mid1];
        infirst = mid1 - 1 - l1;
        inffirst = mid1 - l1;
        insecond = mid2 - 1;
    } else {
        int64_t min, max;
        int64_t min_pos, max_pos;

        if (tmp[l1] > tmp[l2]) {
                min = tmp[l2];
                min_pos = l2;
        } else {
                min = tmp[l1];
                min_pos = l1;
        }
        if (tmp[r1] > tmp[r2]) {
                max = tmp[r1];
                max_pos = r1;
        } else {
                max = tmp[r2];
                max_pos = r2;
        }

        int64_t mediana = (min + max) / 2;
        int64_t answer;
        std::vector<int64_t> length;
        
        MedianaSearch(mediana, tmp, l1, r1, l2, r2, length);
        infirst = length[1];
        inffirst = length[1];
        insecond = length[3];

        total = length[0] + length[2] - 1;
        if (min_pos == l2) {
            answer = length[1] ;
        } else if (min_pos == l1) {
            answer = length[3];
        }
        res[total] = tmp[answer];
    }
    --depth;


    std::thread tt1(Merge, std::ref(tmp), l1, l1 + infirst, l2, insecond , std::ref(res), lres, depth, limiter);
    std::thread tt2(Merge, std::ref(tmp),  l1 + inffirst + 1, r1, insecond + 1, r2, std::ref(res), total + 1 , depth, limiter);
    tt1.join();
    tt2.join();
}


void MergeSortForOne(std::vector<int64_t>& tmp, int64_t l1, int64_t r1, std::vector<int64_t>& res, int64_t depth, bool limiter) {

    if (depth <= 0) {
        OrdinaryMergeSort(tmp, l1, r1, res);
        return;
    }
    if (l1 >= r1) {
        return;
    } else {
        int64_t mid = (l1 + r1) / 2;

        --depth;
        std::thread t1(MergeSortForOne, std::ref(res), l1, mid, std::ref(tmp), depth, limiter);
        std::thread t2(MergeSortForOne, std::ref(res), mid + 1, r1, std::ref(tmp), depth, limiter);
        t1.join();
        t2.join();
        Merge(tmp, l1, mid, mid + 1, r1, res, l1, depth, limiter);
    }
}


void CountingSort(std::vector<int64_t>& array, int64_t n) {
    std::vector<int64_t>::iterator iter = std::max_element(array.begin(), array.end());
    int64_t index = std::distance(array.begin(), iter);
    int64_t k = array[index];
    std::vector<int64_t> second(n);
    std::vector<int64_t> c(k + 1, 0);
 
    for (int64_t i = 0; i < n; ++i) {
        ++c[array[i]];
    }
   
    for (int64_t i = 1; i < k + 1; ++i) {
       c[i] += c[i - 1];
    }
  
    for (int64_t i = n; i > 0; --i) {
        second[c[array[i - 1]] - 1] = array[i - 1];
        --c[array[i - 1]];
    }
    for(size_t i = 0; i < n; ++i) {
        array[i] = second[i];
    }
}


int main (int argc, char const *argv[]) {
    std::vector<int64_t> vec1;
    try {
        if (argc == 2) {
            std::string filename = argv[1];
            std::ifstream file(filename);

            if (!ReadFile(file, vec1)) {
                Reader(vec1);
            }
        } else {
            Reader(vec1);
        }
    }
    catch (const std::exception& e) {
        std::cout << e.what() << std::endl;
    }

    int64_t l1 = 0, r1 = vec1.size() - 1;

    std::vector<int64_t> tmp1_1(vec1.size());
    tmp1_1.assign(vec1.begin(), vec1.end());

    int64_t depth = log2(std::thread::hardware_concurrency());

    bool limiter = (depth == 2) ? false : true;
    
    try {
        std::thread t1(MergeSortForOne, std::ref(tmp1_1), l1, r1, std::ref(vec1), depth, limiter);
        t1.join();
    } catch (std::runtime_error &ex) {
        std::cerr << ex.what();
    }
    

    if (argc == 2) {
        std::string filename = "output";
        std::ofstream file(filename);
        if (file.is_open()) {
            for (int i = 0; i < vec1.size(); ++i) {
                file << vec1[i] << std::endl;
          }
          file << '\n';
        }
        file.close();
    } else {
        for (int i = 0; i < vec1.size(); ++i) {
            std::cout << vec1[i] << ' ';
        }
        std::cout << "\n";
    }
    return 0;
}
\end{lstlisting}

\pagebreak
