\pagebreak
\section{Выводы}
Выполнив курсовой проект по курсу «Дискретный анализ», я приобрела практические и теоретические навыки благодаря полученным в течении курса знаниям. Задачи с применением многопоточности всегда интересные. Они решают две главные задачи:
\begin{enumerate}
	\item Одновременно выполняют несколько действий.
	\item Ускоряют вычисления.
\end{enumerate}
Действительно, многопоточность решает важные задачи, ее использование ускоряет работу программ. Но при неправильном использовании она создает проблем. Существует две проблемы:
\begin{enumerate}
	\item Взаимная блокировка (deadlock). Несколько потоков находятся в ожидании ресурсов, занятых друг другом, и ни один из них не может продолжать выполнение.
	\item Состояние гонки (race condition). Работа программы зависит от того, в каком порядке выполняются части кода.
\end{enumerate}
Потому мало просто придумать параллельный алгоритм, нужно учесть нюансы реальной работы на машинах и предусмотреть возможные проблемы и реализовать их решения.
\pagebreak